\documentclass[a4paper, 14pt]{article}

\usepackage[T2A]{fontenc}
\usepackage[utf8]{inputenc}
\usepackage[english, main=russian]{babel}

\usepackage{tikz} 
\usetikzlibrary{automata, positioning}

\tikzset{
->,
node distance=2.5cm, % specifies the minimum distance between two nodes. Change if necessary.
every state/.style={thick, fill=gray!10}, % sets the properties for each ’state’ node
initial text=$ $, % sets the text that appears on the start arrow
}
%
\begin{document}

\section{Задача 1}

Определить регулярность языка $L = \{w \ \Big\vert \mid w \mid_{aba} = \mid w \mid_{ab}  \& \ w \in \{a, b, c\}^{*} \}$

\subsection{Решение}

\begin{figure}[ht] 
	\centering 

	\begin{tikzpicture}
		\node[state, initial, accepting] (q0) {0};
		\node[state, accepting, right of=q0] (q3) {c};
		\node[state, accepting, above of=q3] (q1) {a};
		\node[state, accepting, below of=q3] (q2) {b};
		\node[state, accepting, right of=q3] (q6) {aba};
		\node[state, accepting, below right of=q6] (q4) {ba};
		\node[state, right of=q1] (q5) {ab};
		
		\draw (q0) edge[above] node{a} (q1)
			  (q0) edge[below] node{b} (q2)
			  (q0) edge[above] node{c} (q3)
			  (q1) edge[loop above] node{a} (q1)
			  (q1) edge[left, bend right] node{c} (q3)
			  (q1) edge[above, bend left] node{b} (q5)
			  (q2) edge[loop below] node{b} (q2)
			  (q2) edge[right, bend right] node{c} (q3)
			  (q2) edge[below] node{a} (q4)
			  (q3) edge[right, bend right] node{a} (q1)
			  (q3) edge[left, bend right] node{b} (q2)
			  (q3) edge[loop right] node{c} (q3)
			  (q4) edge[right, bend right] node{b} (q5)
			  (q4) edge[loop below] node{a} (q4)
			  (q5) edge[left, bend left] node{a} (q6)
			  (q6) edge[left, bend left] node{b} (q5)
			  (q6) edge[below, bend left] node{c} (q3)
			  (q6) edge[loop below] node{a} ();
			  
			  % (q2) (q3)
	\end{tikzpicture}
	\caption{Автомат для языка $L$}
	\label{fig:task_automata}
\end{figure}

Т.к. удалось построить автомат, то язык регулярный.

\end{document}
