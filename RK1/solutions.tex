\documentclass[a4paper, 14pt]{article}

\usepackage[T2A]{fontenc}
\usepackage[utf8]{inputenc}
\usepackage[english, main=russian]{babel}

\usepackage{amsmath}

\usepackage{tikz} 
\usetikzlibrary{automata, positioning}



\usepackage{enumitem}
\setlist[enumerate]{label*=\arabic*.}

\tikzset{
->,
node distance=2.5cm, % specifies the minimum distance between two nodes. Change if necessary.
every state/.style={thick, fill=gray!10}, % sets the properties for each ’state’ node
initial text=$ $, % sets the text that appears on the start arrow
}
%
\begin{document}



\tableofcontents 

\newpage

\section{Задача 1}

Определить регулярность языка $L = \{w \ \Big\vert \mid w \mid_{aba} = \mid w \mid_{ab}  \& \ w \in \{a, b, c\}^{*} \}$

\subsection{Решение}

\begin{figure}[ht] 
	\centering 

	\begin{tikzpicture}
		\node[state, initial, accepting] (q0) {0};
		\node[state, accepting, right of=q0] (q3) {c};
		\node[state, accepting, above of=q3] (q1) {a};
		\node[state, accepting, below of=q3] (q2) {b};
		\node[state, accepting, right of=q3] (q6) {aba};
		\node[state, accepting, below right of=q6] (q4) {ba};
		\node[state, right of=q1] (q5) {ab};
		
		\draw (q0) edge[above] node{a} (q1)
			  (q0) edge[below] node{b} (q2)
			  (q0) edge[above] node{c} (q3)
			  (q1) edge[loop above] node{a} (q1)
			  (q1) edge[left, bend right] node{c} (q3)
			  (q1) edge[above, bend left] node{b} (q5)
			  (q2) edge[loop below] node{b} (q2)
			  (q2) edge[right, bend right] node{c} (q3)
			  (q2) edge[below] node{a} (q4)
			  (q3) edge[right, bend right] node{a} (q1)
			  (q3) edge[left, bend right] node{b} (q2)
			  (q3) edge[loop right] node{c} (q3)
			  (q4) edge[right, bend right] node{b} (q5)
			  (q4) edge[loop below] node{a} (q4)
			  (q4) edge[below, bend left] node{c} (q3)
			  (q5) edge[left, bend left] node{a} (q6)
			  (q6) edge[left, bend left] node{b} (q5)
			  (q6) edge[below, bend left] node{c} (q3)
			  (q6) edge[loop below] node{a} ();
			  
			  % (q2) (q3)
	\end{tikzpicture}
	\caption{Автомат для языка $L$}
	\label{fig:task_automata}
\end{figure}

Автомат построен на основе суффиксов слов. Если ниразу не встретилось подслово $ab$, то такое слово нам подходит. Если встречается суфикс $ab$, то за ним же должен следовать
символ $a$, чтобы уравновесить $ab$ и $aba$, иначе сделать равным количество вхождений $aba$ и $ab$ не получится, т.к. каждое вхождение $aba$ влечет за собой вхождение $ab$. Поэтому 
из состояния $ab$ по $b$, $c$ мы попадаем в ловушку, а по символу $a$ попадаем в конечное состояние $aba$.


\section{Задача 2}
Проанализировать язык истинных выражений, представляющих собой утверждение вида $N_1 + N_2 > N_0$, где $N_0$, $N_1$ и $N_2$ - двоичные числа.

Алфавитом языка является множество: $\{0, 1, +, > \}$. Обозначим данный язык $L$.
Рассмотрим следующей слово $w$, принадлежащее языку $L$:

$$
w =  11^n + 11^n00^n > 11^{2n}0
$$

$$
$$


Здесь $N_1 = 11^n$, $N_2 = 11^n00^n$, $N_3 = 11^{2n}0$.
Пусть задана $n$ - длина накачки. Возьмем описанное вышел слово, длина которого > $n$.
Рассмотрим варианты разбиения этого слова на подслова $u, x, v ,y, w$.


\begin{enumerate}

	\item{
			Если $x$ или $y$ содержат символы +, >, то при любой накачке полученное слово не будет принадлежать языку $L$. Поэтому стоит рассматривать лишь такие разбиения, в которых $x, y \in N_i, \ i=\overline{0,2}$
		}
    \item{
			Пусть $xvy$ лежит в $N_1$. Тогда при отрицательной накачке, полученное слово не будет принадлежать языку $L$, т.к. изначально левая и правая часть отличались на единицу.
		}
	\item{
			Пусть $x$ лежит в $N_1$, а $y$ в $N_2$. Тогда при отрицательной накачке, полученное слово не будет принадлежать языку $L$(аналогично пункту 2).
		}	
	\item{
			Пусть $xvy$ лежит в $N_2$. Тогда при отрицательной накачке, полученное слово не будет принадлежать языку $L$(аналогично пункту 2).
		}
	\item{
			Пусть $xvy$ лежит в $N_0$. Тогда с помощью накачки $N_0$ можно сделать сколь угодно большим и неравенство станет неверным $L$.
		}

	\item{
			Пусть $x \in N_2$, $y \in N_0$, тогда $x$ имеет вид $0^k$, а $y = 1^m$. Необходимо рассмотреть следующие случаи:
			
    \begin{enumerate}
		\item{ $m = k$; Т.к. $x = 0^m$, а $y = 1^m$, то при положительной накачке, полученное слово не будет принадлежать языку $L$, т.к порядки левой и правой частей равны,
			но левая часть была накачана нулями, а правая единицами, и из-за этого число справа будет больше суммы чисел слева, поскольку в некотором разряде $i$ в числе $N_0$ будет 1, а в сумме $N_1 + N_2$ в том же разряде будет стоять 0. 
			} 
		\item{ $m > k$; При положительной накачке, число $N_0$ будет расти быстрее, чем $N_2$, поэтому существует длина накачки $i$, при которой порядок $N_0$ станет больше, чем $N_1 + N_2$, и слово не будет принадлежать языку.} 
		\item{ $m < k$; Если k - m > 1, выполним отрицательную накачку и при сложении чисел слева, разряд суммы будет меньше чем разряд числа справа, т.к. при сложении число может увеличиться лишь на один разряд. Есл m + 1 = k, то при отрицательной накачке порядки чисел слева и справа будут равны, но число $N_0$ будет больше, т.к. при солжении $N_1$ и $N_2$ в получившемся числе на позиции $i$ появится 0, причем $i > 0$. Т.о. неравенство не будет верным} 
    \end{enumerate}
		}
\end{enumerate}

Таким образом для любого n можно подобрать слово, которое не накачивается, значит данный в условии язык не является контекстно-свободным.


\section{Задача 3}

Определить, описывает ли данная грамматика регулярный язык 

\begin{equation*}
	\begin{aligned}[t]
		S &\rightarrow TSTa\\  
		S &\rightarrow SS\\
		S &\rightarrow bb
	\end{aligned}
	\qquad \ \qquad
	\begin{aligned}[t]
		T &\rightarrow a\\
		T &\rightarrow b\\
		T &\rightarrow TT\\
	\end{aligned}
\end{equation*}

\newpage

\subsection{Решение}

Язык нетерминала $T$ совпадает с языком задаваемым регулярными выражением $(a|b)^{+}$.
Промежуточное представление для $S$ можно записать следующим образом:

\begin{equation*}
	\begin{aligned}[t]
		S &\rightarrow (a|b)^{+}S(a|b)^{+}a\\  
		S &\rightarrow SS\\
		S &\rightarrow bb
	\end{aligned}
\end{equation*}


Гипотеза: Язык нетерминала $S$ описывается регулярным выражением

$$(((a|b)^{+}(bb)^{*})^{+}((a|b)^{+}a)^{+} | (bb)^{+})^{+}$$

\section{Задача 4}

Пусть $h(w)$ - слово, получающееся из $w$ удвоением каждой буквы. Например, $h(aba^2) = a^{2}b^{2}a^{4}$. Запишем эти слова друг под другом так, чтобы первые буквы $w$ и $h(w)$ образовали пару, вторые - следующую за ней, и т.д. Недостающуую длину в $w$ дополним "решетками".

Исследовать язык пар слов $(w, h(w))$, поступающих на вход анализатора разбитыми таким образом на пары букв, т.е. поступающих параллельно (т.е. элементы входного алфавита - вектора
$
\begin{pmatrix}
	w_i\\
	v_i
\end{pmatrix}
$ 
, где $w_i, v_i \in \{a, b, \# \}$
). 

В нашем примере вход анализатора будет представлять собой следующую последовательность пар:

$$
\begin{pmatrix}
	a\\
	a
\end{pmatrix}
\begin{pmatrix}
	b\\
	a
\end{pmatrix}
\begin{pmatrix}
	a\\
	b	
\end{pmatrix}
\begin{pmatrix}
	a\\
	b	
\end{pmatrix}
\begin{pmatrix}
	\#\\
	a	
\end{pmatrix}
\begin{pmatrix}
	\#\\
	a	
\end{pmatrix}
\begin{pmatrix}
	\#\\
	a	
\end{pmatrix}
\begin{pmatrix}
	\#\\
	a	
\end{pmatrix}
$$

\end{document}
