\documentclass[a4paper, 14pt]{article}

\usepackage[T2A]{fontenc}
\usepackage[utf8]{inputenc}
\usepackage[english, main=russian]{babel}

\usepackage{amsmath}

\usepackage{tikz} 
\usetikzlibrary{automata, positioning}

\usepackage{mathtools}

\usepackage{enumitem}
\setlist[enumerate]{label*=\arabic*.}

\tikzset{
->,
node distance=2.5cm, % specifies the minimum distance between two nodes. Change if necessary.
every state/.style={thick, fill=gray!10}, % sets the properties for each ’state’ node
initial text=$ $, % sets the text that appears on the start arrow
}
%
\begin{document}

\begin{titlepage}
    \begin{center}
        \vspace*{1cm}
            
        \Huge
        \textbf{Теория формальных языков. Рубежный контроль №2}
		
        \vspace{0.5cm}
        \LARGE
        Вариант №23
            
        \vspace{1.5cm}
            
        \textbf{Киселев Кирилл}
            
        \vfill
            
        \vspace{0.8cm}
		
        \Large
        Теоретическая информатика и компьютерные технологии\\
        МГТУ им. Н.Э. Баумана\\
        декабрь 2023
            
    \end{center}
\end{titlepage}

\newpage

\tableofcontents 

\newpage

\section{Задача 1}

Язык SRS $a \rightarrow bab$, $a^3 \rightarrow a^2$, $ba \rightarrow ac$ над множеством базисных слов $b^n a^n$

\subsection{Решение}


Замечания:

\begin{enumerate}
  \item Любое не пустое слово содержит хотя бы одну $a$
  \item Буквы $a$ могут только уменьшаться
  \item $|w|_a \leq |w|_b + |w|_c$
  \item Можно бесконечно двигать влево самую первую букву $a$
  \item Слова могут начинаться только с $b^i a^k$, где $k > 0$
  \item Применение правила 3 ограничивает сдвиг тех $a$, которые находились правее буквы $a$, к которой было применено правило.
  \item При появлении ол
  


    
\end{enumerate}

$$w = b^{i_0}a^{p_1}c^{k_1}b^{i_1}a^{p_2}c^{k_2}b^{i_2}a^{p_3}c^{k_3}b^{i_3}\ldots a^{p_r}c^{k_r}b^{i_r}a^{p_{r + 1}}, p_1 \geq 1 $$

\[ \sum_{m=1}^{r + 1}{p_m} \leq \sum_{m=0}^{r}{i_m} +  \sum_{m=1}^{r}{k_m}\]

\newpage 
\section{Задача 2}

Язык $\Big\{ w \ \Big| \ |w|_{ab} \ = \ |w|_{baa} \ \& \ w = w^{R} \Big\}$. Алфавит $\{a, b\}$


\subsection{Решение}

Пусть $L_1 = \{w \ \Big| \ |w|_{ab} \ = \ |w|_{baa} \}$, $L_2 = \{w \ \Big| w = w^{R}\}$. Язык $L_1$ регулярный, а язык $L_2$ контекстно-свободный. Значит исходный язык $L$ является КС, как пересечение КС и регулярного языков.

Докажем недетерминированность $L$. Пусть $n$ - длина накачки, положим $k = n + 1$. Тогда возьмем следующие слова: 

$$w_1 = a^{2k}b^{2k}a^{2k}, w_2 = a^{2k}b^{2k}a^{3k}b^{2k}a^{3k}b^{2k}a^{2k}$$

Пусть $x = a^{2k}b^{2k}a^{2k - 1}$, $y = a$, $z = a^{k + 1}b^{2k}a^{3k}b^{2k}a^{2k}$. Необходимо рассмотреть 2 случая:

\begin{enumerate}
  \item{Рассмотрим общий перефикс $x$. Пусть $x = x_0 x_1 x_2 x_3 x_4$. В случаях: $x_1 = a^k$ и $x_3 = a^p$, $x_1 = a^k$ и $x_3 = b^p$, $x_1 = b^k$ и $x_3 = b^p$}; отрицательная накачка в $w_2$ выводит слово из языка, т.к полученное слово уже не будет являться палиндромом. Если $x_1 = a^{k_1} b^{k_2}$, либо $x_2 = a^{k_1} b^{k_2}$, то отрицательная накачка выводит оба слова из языка
  \item{Пусть $x = x_0 x_1 x_2$, $y = y_0 y_1 y_2$, $z = z_0 z_1 z_2$}. Т.к по условию леммы $|x_1 x_2| \leq n$, то $x_1 = a^{k_1}$ и $x_2 = a^{k_2}$, $k_1 + k_2 \leq n$, $k_1 > 0$. Также $y_1$ равно либо пустому слову, либо $a$, тогда слово $x_0 x_1^{i} x_2 y_0 y_1^{i} y_2$ при любом $i \neq 1$ не принадлежит $L$, т.к. не является палиндромом.
\end{enumerate}

Следовательно, данный язык не является детерменированным КС языком.

\newpage 

\section{Задача 3}

Язык атрибутной грамматики для регулярок:

$$
\begin{aligned}
  &[S] \rightarrow [Regexp] &;  \quad  \\
  &[Regexp] \rightarrow [Regexp][Regexp] &; \quad  &{Regexp_{1}.val \neq \varepsilon, Regexp_{2}.val \neq \varepsilon} \\
  & & \quad &{Regexp_{0}.val \coloneqq Regexp_{1}.val ++ Regexp_{2}.val}\\
  &[Regexp] \rightarrow ([Regexp]|[Regexp]) &; \quad &{Regexp_{1}.val \neq \varepsilon \lor Regexp_{2}.val \neq \varepsilon}, \\
  & & \quad &{Regexp_{1}.val \neq |}, Regexp_{0}.val \coloneqq | \\
  &[Regexp] \rightarrow ([Regexp])* &; \quad &{Regexp_{1}.val \neq \varepsilon}\\
  & & \quad &{Regexp_{1}.val \neq *}, Regexp_{0}.val \coloneqq * \\
  &[Regexp] \rightarrow \varepsilon &; \quad &{Regexp.val \coloneqq \varepsilon} \\
  &[Regexp] \rightarrow a &;  \quad &{Regexp.val \coloneqq a} \\
  &[Regexp] \rightarrow b &; \quad  &{Regexp.val \coloneqq b}
\end{aligned}
$$



\subsection{Решение}

Рассмотрим подвыражения, которые запрещены согласно ограничениям налагаемым условиями на аттрибут:

\begin{enumerate}
  \item{$(\varepsilon | \varepsilon)$}
  \item{$((\cdot \ | \ \cdot )\ | \ \cdot)$}
  \item{$(\varepsilon)*$}
  \item{$((\cdot)*)*$}
\end{enumerate}

Для исключения подслов вида 1, 2 введем три новых нетерминала, и вместо правила $[Regexp] \rightarrow ([Regexp]|[Regexp])$ введем правило $[Regexp] \rightarrow [Regexp']$

$$
\begin{aligned}
  &[Regexp'] &\rightarrow \quad &{(\varepsilon|[Regexp_{rhs}])} & &[Regexp_{lhs}] &\rightarrow \quad &{[Regexp_{lhs}][Regexp_{lhs}]} \\
  &[Regexp'] &\rightarrow \quad &{([Regexp_{lhs}]|\varepsilon)} & &[Regexp_{lhs}] &\rightarrow \quad  &{[Regexp_{iter}]} \\
  &[Regexp_{rhs}] &\rightarrow \quad &{[Regexp']} & &[Regexp_{lhs}] &\rightarrow \quad  &a\\ 
  &[Regexp_{rhs}] &\rightarrow \quad &{[Regexp_{iter}]} & &[Regexp_{lhs}] &\rightarrow \quad  &b\\ 
  &[Regexp_{rhs}] &\rightarrow \quad &{[Regexp_{rhs}][Regexp_{rhs}]} \\
  &[Regexp_{rhs}] &\rightarrow \quad &a \\
  &[Regexp_{rhs}] &\rightarrow \quad &b
\end{aligned}
$$





Для того, чтобы исключить выражения вида 3, 4 введем новый нетерминал $Regexp_{iter}$
$$
\begin{aligned}
  [Regexp_{iter}] &\rightarrow [Regexp'] \\
  [Regexp_{iter}] &\rightarrow [Regexp_{iter}][Regexp_{iter}] \\
  [Regexp_{iter}] &\rightarrow a \\
  [Regexp_{iter}] &\rightarrow b
\end{aligned}
$$

\end{document}
