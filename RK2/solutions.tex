\documentclass[a4paper, 14pt]{article}
\usepackage[T2A]{fontenc} \usepackage[utf8]{inputenc}
\usepackage[english, main=russian]{babel}

\usepackage{amsmath}
\usepackage{amsfonts}
\usepackage{tikz} 
\usetikzlibrary{automata, positioning, arrows}
\usepackage{mathtools}
\usepackage{graphicx}
\usepackage{float}
\usepackage{enumitem}
\setlist[enumerate]{label*=\arabic*.}

\tikzset{
->,
>=stealth',
node distance=4.5cm, % specifies the minimum distance between two nodes. Change if necessary.
every state/.style={thick, fill=gray!10}, % sets the properties for each ’state’ node
initial text=$ $, % sets the text that appears on the start arrow
}
%
\begin{document}

\begin{titlepage}
    \begin{center}
        \vspace*{1cm}
            
        \Huge
        \textbf{Теория формальных языков. Рубежный контроль №2}
		
        \vspace{0.5cm}
        \LARGE
        Вариант №23
            
        \vspace{1.5cm}
            
        \textbf{Киселев Кирилл}
            
        \vfill
            
        \vspace{0.8cm}
		
        \Large
        Теоретическая информатика и компьютерные технологии\\
        МГТУ им. Н.Э. Баумана\\
        декабрь 2023
            
    \end{center}
\end{titlepage}

\newpage

\tableofcontents 

\newpage

\section{Задача 1}

Язык SRS $a \rightarrow bab$, $a^3 \rightarrow a^2$, $ba \rightarrow ac$ над множеством базисных слов $b^n a^n$

\subsection{Решение}


Замечания:

\begin{enumerate}
  \item Любое не пустое слово содержит хотя бы одну $a$
  \item Буквы $a$ могут только уменьшаться
  \item $|w|_a \leq |w|_b + |w|_c$
  \item Можно бесконечно двигать влево самую первую букву $a$
  \item Слова могут начинаться только с $b^i a^k$, где $k > 0$
  \item Применение правила 3 ограничивает сдвиг тех $a$, которые находились правее буквы $a$, к которой было применено правило.

    
Начальное количество букв $a$ и $b$ равно, далее количество исходных $b$ может либо остаться таким же, либо с помощью правила 3 некоторые $b$, могут перейти в $c$, но суммарное количество $b$ и $c$ останется равным $n$. Буквы $a$ могут быть переписаны с помощью правила 2, и их количество станет меньше $n$. Каждой буквой $a$ с помощью первого и третьего правил могут порождаться некоторые блоки, в которых количество букв $b$ и $c$ сбалансированно, т.к. правило 1 порождает блоки вида $b^k a b^k$, в которых количество букв $b$ сбалансированно. Поэтому при построении PDA стоит ввести два стековых символа, первый будет использовать для подсчета исходных $b$ и букв $c$, полученных из этих букв $b$. Второй символ необходим для проверки баланса букв $b$ и $c$, которые были пораждены с помощью повторения правил 2, 3. Также этот блок может содержать один внутренний блок такой же структуры, из-за наличия правила 3, тогда исходный блок должен заканчиваться на $c^r$, после чтения которых в стеке должны оставаться лишь символы первого вида.



% b^3n a b^n b^2n c
% b^3n a b^n b^2n c - это слово не накачивается во втором случае, если брать суффикс b^3n a b^n 
% b^3n a b^n b^n a b^2n a^3n-2   - это слово не накачивается во втором случае, если брать суффикс b^3n a b^n 
% b^2n a c^2n a^4n-1 

% b^2n b^n a b^n a
% b^2n a c^n  b^n a


% a c
% b^4n a a
% b^2n a c^2n a
% b^3n a b^3n c 

\end{enumerate}


Докажем, что данный язык не принадлежит классу $\mathtt{DCFL}$. Рассмотрим следующие слова:

% b^3n a b^n b^2n c - это слово не накачивается во втором случае, если брать суффикс b^3n a b^n 
% b^3n a b^n b^n a b^2n a^3n-2   - это слово не накачивается во втором случае, если брать суффикс b^3n a b^n 
%
\begin{itemize}
  \item $w_1 = b^{3n} a b^n b^{2n} c$
  \item $w_2 = b^{3n} a b^{n} b^{n} a b^{2n} c^{2n} a^{n-2}$
  % \item $w_2 = b^{3n} a b^{2n} a b^{2n} a^{3n-2}$
\end{itemize}

% Для получение слова $w_1$ небходимо к слову $ba$ применить сначала правило 3, далее $3n$ раз правило 1. Слово $w_2$ получается из базового слова $b^{3n}a^{3n}$ применением ко второй букве $a$ правила 1 $2n$ раз. Параметр $n$ - это длина накачки, считаем, что $n$ больше нуля, иначе сделаем замену $n = n + 1$. Пусть $x = b^{3n} a b^n$, $y = b^{2n} c$, $z = b^{n} a b^{2n} a^{3n-2}$.
Рассмотрим два случая:

\begin{enumerate}
  \item{Накачка общего префикса $x$. Пусть $x = x_0 x_1 x_2 x_3 x_4$.Рассмотрим слово $w_2$. Если $x_1 x_2 x_3 \in b^{3n}$, то отрицательная накачка выводит слово из языка, теряется баланс с буквами $c$. Если фрагмент $x_1 x_2 x_3 \in b^{n}$ и располагается справа от первой буквы $a$, то любая($i \ne 1$) накачка выводит слово из языка}, т.к. теряется баланс между буквами $b$, которые были пораждены с помощью первого правила. Если $x_1 \in b^{3n}$, $x_3 \in b^{n}$, то отрицательная накачка выводит слово из языка, т.к. нарушаются обе указанных выше причины. Пусть $x_1 = b^{k_1} a b^{k_2}$, $x_2 = b^{k_3}$ и $i = 2$. Получим слово: $b^{n} b^{2n} a [b^{k_2} b^{k_1} a b^{2n + k_3} a b^{2n} c^{2n}] a^{n-2}$. Подслово выделенное квадратными нарушает баланс между $b$ и $c$, т.к. $k_1 + k_2 < n < 2n$. Аналогично для случая  $x_2 = b^{k_1} a b^{k_2}$, $x_1 = b^{k_3}$ 

    % Если $a \in x_1$ либо $a \in x_3$, то накачка при $i > 1$ выводит слово из языка, т.к. в блоке $b^{3n} \dots c^{2n}$ не может быть больше двух букв $a$, т.к. наличие букв $c$ говорит о применение правила 3, тогда если оно было применено к исходным буквам $b$, то в этом блоке должна находиться лишь одна буква $a$, если правило 3 было применено к буквам порожденным с первого правила, то в таком блоке должно быть две буквы $a$.
  \item{Синхронная накачка. Рассмотрим слово $w_1$. Пусть $x = x_0 x_1 x_2$, $y = y_0 y_1 y_2$. Т.к по условию леммы $|x_1 x_2| \leq n$, то $x_1 = b^{k_1}$ и $x_2 = b^{k_2}$, $k_1 + k_2 \leq n$, $k_1 > 0$}. Если $y_1 = b^{k_3}, k_3 >= 0$, то накачка при $i > 1$ выводит слово из языка, т.к. теряется баланс между буквами $b$, порожденными первым правилом. Если $y_1 = c$, то накачка при $i > 2$ выводит слово из языка, т.к максимальная длина суффикса состоящего из букв $c$ в данном случае может быть только единицей.
    Пусть $y_1 = b^{k_3}c, k_3 > 0$, тогда при $i = 0$ получим слово $w_3 = b^{3n}ab^{3n - k_1 - k_3}$, которое не принадлежит языку, потому что оно могло быть получено только из базового слова $ba$, но т.к. $k_1 + k_3 \geq 2$, то слово из которого получено $w_3$ должно быть $b^{k_4}a, k_4 \geq 2$, а это слово не является базовым. 
\end{enumerate}

Таким образом, данный язык не принадлежит классу $\mathtt{DCFL}$. 


\newpage

\begin{figure}[H]
	\centering 

    \rotatebox{90}{
	\begin{tikzpicture}
		\node[state, initial] (q0) {0};
		\node[state, right of=q0] (q1) {1};
		\node[state, right of=q1] (q2) {2};
		\node[state, right of=q2] (q3) {3};
		\node[state, below left of=q3] (q4) {4};
		\node[state, right of=q4] (q5) {5};
		\node[state, right of=q5] (q6) {6};
		\node[state, below left of=q2] (q7) {7};
	  \node[state, below of=q5] (q13) {13};
		\node[state, below left of=q13] (q10) {10};
    \node[state, left of=q10] (q9) {9};
  	\node[state, left of=q9] (q8) {8};
		\node[state, right of=q10] (q11) {11};
		\node[state, below left of=q7, accepting] (q12) {12};
		\node[state, right of=q13, accepting] (q14) {14};
		\node[state, below left of=q10, accepting] (q15) {15};
		\node[state, below right of=q0] (q16) {16};
		  
    \draw
      (q0) edge[below] node{$\varepsilon$} (q1)
      (q1) edge[loop above] node{$b,\frac{\forall}{X\forall}$} (q1)
      (q1) edge[above] node{$a,\frac{X(Z_0)}{X(Z_0)},\frac{Y}{\varepsilon}$} (q2)
      
      (q2) edge[loop above] node{$b,\frac{X}{\varepsilon}$} (q2)
      
      (q2) edge[pos=0.8, left, bend right] node{$\varepsilon,\frac{Y}{\varepsilon}$} (q16)
      (q2) edge[above] node{$c,\frac{X}{XX}$} (q3)
      (q2) edge[left] node{$\varepsilon,\frac{X}{X}$} (q4)
      (q3) edge[right] node{$\varepsilon,\frac{X}{X}$} (q4)
      
      (q3) edge[below, bend left] node{$b,\frac{X}{\varepsilon}$} (q2)
      
      (q3) edge[loop above] node{$c,\frac{X}{XX}$} (q3)
      
      (q5) edge[above] node{$c,\frac{X}{XX}$} (q6)
      (q5) edge[left] node{$c,\frac{X}{\varepsilon}$} (q13)
      (q6) edge[below, bend left] node{$b,\frac{X}{\varepsilon}$} (q5)
      (q7) edge[loop above] node{$c,\frac{Y}{YY}$} (q7)
      (q8) edge[loop below] node{$a,\frac{Y}{\varepsilon}$} (q8)
      (q13) edge[loop below, in=330, out=300, looseness=5] node{$c,\frac{X}{\varepsilon}$} (q13)
      (q10) edge[right] node{$\varepsilon,\frac{Y}{\varepsilon}$} (q15)
      
      (q13) edge[above] node{$\varepsilon,\frac{Y}{\varepsilon}$} (q14)
      (q13) edge[above, bend right] node{$\varepsilon,\frac{Y}{Y}$} (q8)
      (q7) edge[left] node{$a,\frac{Y}{Y}$} (q8)
      (q2) edge[right] node{$a,\frac{Y}{\varepsilon}$} (q8)
      
      (q4) edge[above] node{$a,\frac{X}{X}$} (q5)
      (q2) edge[pos=0.3, left] node{$c,\frac{Y/Z_0}{Y/Z_0}$} (q7)

      
      (q8) edge[below] node{$\varepsilon, \frac{Y}{Y}$} (q9)
      
      (q4) edge[loop left] node{$b,\frac{X}{XX}$} (q4)
      (q9) edge[loop above] node{$b,\frac{X}{XX}$} (q9)
      (q9) edge[below] node{$a, \frac{Y}{Y}, \frac{X}{X}$} (q10)
      % (q10) edge[above, bend right] node{$b, \frac{Y}{XY}$} (q9)
      
      (q10) edge[pos=0.8,left] node{$\varepsilon,\frac{X}{X}$} (q4)
      (q11) edge[pos=0.7, right] node{$\varepsilon,\frac{X}{X}$} (q4)
      (q10) edge[above] node{$c,\frac{X}{XX}$} (q11)
      (q11) edge[below, bend left] node{$b,\frac{X}{\varepsilon}$} (q10)
      (q11) edge[loop below] node{$c,\frac{X}{XX}$} (q11)
      (q6) edge[loop above] node{$c,\frac{X}{XX}$} (q6)
      (q5) edge[loop above] node{$b,\frac{X}{\varepsilon}$} (q5)
      (q10) edge[loop below] node{$b,\frac{X}{\varepsilon}$} (q10)
      (q0) edge[left] node{$\varepsilon, \frac{Z_0}{Z_0}$} (q12)
      (q7) edge[pos=0.3, left] node{$\varepsilon, \frac{Z_0}{Z_0}$} (q12)
      (q10) edge[below, bend left] node{$\varepsilon, \frac{Y}{\varepsilon}$} (q8)
      (q8) edge[left] node{$\varepsilon, \frac{Y}{\varepsilon}, \frac{Z_0}{Z_0}$} (q12)
      (q16) edge[pos=0.3, left] node{$\varepsilon, \frac{Z_0}{Z_0}$} (q12)
      
      (q0) edge[loop above] node{$b,\frac{\forall}{Y\forall}$} (q0);
      
      
	\end{tikzpicture}
    }
  
	\caption{Автомат для языка $L$}
	\label{fig:task_automata}
\end{figure}


\newpage 

\section{Задача 2}

Язык $\Big\{ w \ \Big| \ |w|_{ab} \ = \ |w|_{baa} \ \& \ w = w^{R} \Big\}$. Алфавит $\{a, b\}$


\subsection{Решение}

Пусть $L_1 = \{w \ \Big| \ |w|_{ab} \ = \ |w|_{baa} \}$, $L_2 = \{w \ \Big| w = w^{R}\}$. Язык $L_1$ регулярный, а язык $L_2$ контекстно-свободный. Значит исходный язык $L$ является КС, как пересечение КС и регулярного языков.

Докажем недетерминированность $L$. Пусть $n$ - длина накачки, положим $k = n + 1$. Тогда возьмем следующие слова: 

$$w_1 = a^{2k}b^{2k}a^{2k}, w_2 = a^{2k}b^{2k}a^{3k}b^{2k}a^{3k}b^{2k}a^{2k}$$

Пусть $x = a^{2k}b^{2k}a^{2k - 1}$, $y = a$, $z = a^{k + 1}b^{2k}a^{3k}b^{2k}a^{2k}$. Необходимо рассмотреть 2 случая:

\begin{enumerate}
  \item{Рассмотрим общий перефикс $x$. Пусть $x = x_0 x_1 x_2 x_3 x_4$. В случаях: $x_1 = a^k$ и $x_3 = a^p$, $x_1 = a^k$ и $x_3 = b^p$, $x_1 = b^k$ и $x_3 = b^p$}; отрицательная накачка в $w_2$ выводит слово из языка, т.к полученное слово уже не будет являться палиндромом. Если $x_1 = a^{k_1} b^{k_2}$, либо $x_2 = a^{k_1} b^{k_2}$, то отрицательная накачка выводит оба слова из языка
  \item{Пусть $x = x_0 x_1 x_2$, $y = y_0 y_1 y_2$, $z = z_0 z_1 z_2$}. Т.к по условию леммы $|x_1 x_2| \leq n$, то $x_1 = a^{k_1}$ и $x_2 = a^{k_2}$, $k_1 + k_2 \leq n$, $k_1 > 0$. Также $y_1$ равно либо пустому слову, либо $a$, тогда слово $x_0 x_1^{i} x_2 y_0 y_1^{i} y_2$ при любом $i \neq 1$ не принадлежит $L$, т.к. не является палиндромом.
\end{enumerate}

Следовательно, данный язык не является детерменированным КС языком.

\newpage 

\section{Задача 3}

Язык атрибутной грамматики для регулярок:

$$
\begin{aligned}
  &[S] \rightarrow [Regexp] &;  \quad  \\
  &[Regexp] \rightarrow [Regexp][Regexp] &; \quad  &{Regexp_{1}.val \neq \varepsilon, Regexp_{2}.val \neq \varepsilon} \\
  & & \quad &{Regexp_{0}.val \coloneqq Regexp_{1}.val ++ Regexp_{2}.val}\\
  &[Regexp] \rightarrow ([Regexp]|[Regexp]) &; \quad &{Regexp_{1}.val \neq \varepsilon \lor Regexp_{2}.val \neq \varepsilon}, \\
  & & \quad &{Regexp_{1}.val \neq |}, Regexp_{0}.val \coloneqq | \\
  &[Regexp] \rightarrow ([Regexp])* &; \quad &{Regexp_{1}.val \neq \varepsilon}\\
  & & \quad &{Regexp_{1}.val \neq *}, Regexp_{0}.val \coloneqq * \\
  &[Regexp] \rightarrow \varepsilon &; \quad &{Regexp.val \coloneqq \varepsilon} \\
  &[Regexp] \rightarrow a &;  \quad &{Regexp.val \coloneqq a} \\
  &[Regexp] \rightarrow b &; \quad  &{Regexp.val \coloneqq b}
\end{aligned}
$$




\subsection{Решение}

Рассмотрим подвыражения, которые запрещены согласно ограничениям налагаемым условиями на аттрибут:

\begin{enumerate}
  \item{$(\varepsilon | \varepsilon)$}
  \item{$((\cdot \ | \ \cdot )\ | \ \cdot)$}
  \item{$(\varepsilon)*$}
  \item{$((\cdot)*)*$}
\end{enumerate}

Для исключения подслов вида 1, 2 введем три новых нетерминала:
$$
\begin{aligned}
  &[Regexp'] &\rightarrow \quad &{(\varepsilon|[Regexp_{rhs}])} & &[Regexp_{lhs}] &\rightarrow \quad &{[Regexp_{lhs}][Regexp_{lhs}]} \\
  &[Regexp'] &\rightarrow \quad &{([Regexp_{lhs}]|\varepsilon)} & &[Regexp_{lhs}] &\rightarrow \quad  &{([Regexp_{iter}])*} \\
  &[Regexp'] &\rightarrow \quad &{([Regexp_{lhs}]|[Regexp_{rhs}])} & &[Regexp_{lhs}] &\rightarrow \quad  &a\\ 
  &[Regexp_{rhs}] &\rightarrow \quad &{[Regexp']}  & &[Regexp_{lhs}] &\rightarrow \quad  &b\\ 
  &[Regexp_{rhs}] &\rightarrow \quad &{([Regexp_{iter}])*} \\
  &[Regexp_{rhs}] &\rightarrow \quad &{[Regexp_{rhs}][Regexp_{rhs}]} \\
  &[Regexp_{rhs}] &\rightarrow \quad &a \\
  &[Regexp_{rhs}] &\rightarrow \quad &b
\end{aligned}
$$


Для того, чтобы исключить выражения вида 3, 4 введем новый нетерминал $Regexp_{iter}$
$$
\begin{aligned}
  [Regexp_{iter}] &\rightarrow [Regexp'] \\
  [Regexp_{iter}] &\rightarrow [Regexp_{iter}][Regexp_{iter}] \\
  [Regexp_{iter}] &\rightarrow a \\
  [Regexp_{iter}] &\rightarrow b
\end{aligned}
$$


В итоге получим следующую грамматику описывающий язык данной атрибутной грамматики: 

$$
\begin{aligned}
  &[S] &\rightarrow \quad &{[Regexp]} & &[Regexp_{iter}] &\rightarrow \quad  &[Regexp'] \\
  &[Regexp] &\rightarrow \quad &{[Regexp][Regexp]} & &[Regexp_{iter}] &\rightarrow \quad &[Regexp_{iter}][Regexp_{iter}] \\ 
  &[Regexp] &\rightarrow \quad &{[Regexp']} & &[Regexp_{iter}] &\rightarrow \quad &a \\
  &[Regexp] &\rightarrow \quad &{([Regexp_{iter}])*} & &[Regexp_{iter}] &\rightarrow \quad &b\\
  &[Regexp'] &\rightarrow \quad &{(\varepsilon|[Regexp_{rhs}])} & &[Regexp_{lhs}] &\rightarrow \quad &{[Regexp_{lhs}][Regexp_{lhs}]} \\
  &[Regexp'] &\rightarrow \quad &{([Regexp_{lhs}]|\varepsilon)} & &[Regexp_{lhs}] &\rightarrow \quad  &{([Regexp_{iter}])*} \\
  &[Regexp'] &\rightarrow \quad &{([Regexp_{lhs}]|[Regexp_{rhs})} & &[Regexp_{lhs}] &\rightarrow \quad  &a \\
  &[Regexp_{rhs}] &\rightarrow \quad &{[Regexp']} & &[Regexp_{lhs}] &\rightarrow \quad  &b \\ 
  &[Regexp_{rhs}] &\rightarrow \quad &{([Regexp_{iter}])*} \\ 
  &[Regexp_{rhs}] &\rightarrow \quad &{[Regexp_{rhs}][Regexp_{rhs}]} \\
  &[Regexp_{rhs}] &\rightarrow \quad &a \\
  &[Regexp_{rhs}] &\rightarrow \quad &b \\
  &[Regexp] &\rightarrow \quad &\varepsilon \\
  &[Regexp] &\rightarrow \quad &a \\
  &[Regexp] &\rightarrow \quad &b 
\end{aligned}
$$



\end{document}
