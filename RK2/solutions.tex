\documentclass[a4paper, 14pt]{article}

\usepackage[T2A]{fontenc}
\usepackage[utf8]{inputenc}
\usepackage[english, main=russian]{babel}

\usepackage{amsmath}

\usepackage{tikz} 
\usetikzlibrary{automata, positioning}



\usepackage{enumitem}
\setlist[enumerate]{label*=\arabic*.}

\tikzset{
->,
node distance=2.5cm, % specifies the minimum distance between two nodes. Change if necessary.
every state/.style={thick, fill=gray!10}, % sets the properties for each ’state’ node
initial text=$ $, % sets the text that appears on the start arrow
}
%
\begin{document}

\begin{titlepage}
    \begin{center}
        \vspace*{1cm}
            
        \Huge
        \textbf{Теория формальных языков. Рубежный контроль №2}
		
        \vspace{0.5cm}
        \LARGE
        Вариант №23
            
        \vspace{1.5cm}
            
        \textbf{Киселев Кирилл}
            
        \vfill
            
        \vspace{0.8cm}
		
        \Large
        Теоретическая информатика и компьютерные технологии\\
        МГТУ им. Н.Э. Баумана\\
        декабрь 2023
            
    \end{center}
\end{titlepage}

\newpage

\tableofcontents 

\newpage

\section{Задача 1}

Язык SRS $a \rightarrow bab$, $a^3 \rightarrow a^2$, $ba \rightarrow ac$ над множеством базисных слов $b^n a^n$

\subsection{Решение}



\newpage 
\section{Задача 2}

Язык $\Big\{ w \ \Big| \ |w|_{ab} \ = \ |w|_{baa} \ \& \ w = w^{R} \Big\}$. Алфавит $\{a, b\}$


\subsection{Решение}

Пусть $L_1 = \{w \ \Big| \ |w|_{ab} \ = \ |w|_{baa} \}$, $L_2 = \{w \ \Big| w = w^{R}\}$. Язык $L_1$ регулярный, а язык $L_2$ контекстно-свободный. Значит исходный язык $L$ является КС, как пересечение КС и регулярного языков.

Докажем недетерминированность $L$. Пусть $n$ - длина накачки. Тогда возьмем следующие слова: $w_1 = a^{2n}b^{2n}a^{2n}$, $w_2 = a^{2n}b^{2n}baabb^{2n}a^{2n}$. Пусть $x = a^{2n}b^{2n}$, $y = a^{2n}$, $z = baabb^{2n}a^{2n}$. Необходимо рассмотреть 2 случая:

\begin{enumerate}
  \item{Рассмотрим общий перефикс $x$. Пусть $x = x_0 x_1 x_2 x_3 x_4$. Если $x_1 = a^k$ и $x_3 = a^p$, либо $x_1 = a^k$ и $x_3 = b^p$, то отрицательная накачка выводит оба слова из языка, т.к полученные слова уже не будут являться палиндромами. Если $x_1 = b^k$ и $x_3 = b^p$}, то отрицательная накачка в $w_2$ выводит слово из языка, т.к. полученное слово не будет являться палиндромом. Если $x_1 = a^{k_1} b^{k_2}$, либо $x_2 = a^{k_1} b^{k_2}$, то отрицательная накачка выводит оба слова из языка
  \item{Пусть $x = x_0 x_1 x_2$, $y = y_0 y_1 y_2$, $z = z_0 z_1 z_2$}. Т.к по условию леммы $|x_1 x_2| \leq n$, то $x_1 = b^{k_1}$ и $x_2 = b^{k_2}$, $k_1 + k_2 \leq n$, $k_1 > 0$. Также $y_1$ в любом случае равно $a^{k_3}$, тогда слово $x_0 x_1^{i} x_2 y_0 y_1^{i} y_2$ при любом $i \neq 1$ не принадлежит $L$, т.к. не является палиндромом.
\end{enumerate}

Следовательно, данный язык не является детерменированным КС языком.

\newpage 

\section{Задача 3}

Язык атрибутной грамматики для регулярок:

\subsection{Решение}

\end{document}
